% This is lnbip.tex the demonstration file of the LaTeX macro package for
% Lecture Notes in Business Information Processing from Springer-Verlag.
% It serves as a template for authors as well.
% version 1.0 for LaTeX2e
%
\documentclass[lnbip]{svmultln}
%
\usepackage{makeidx}  % allows for indexgeneration
% \makeindex          % be prepared for an author index
%

\usepackage{url}
\urldef{\mailsa}\path|lucas.wiener@evry.com, tomase@kth.se, phaller@kth.se|

\newcommand{\code}[1]{\texttt{#1}}
\newcommand{\elq}{ELQ}
\newcommand{\comment}[1]{}

\begin{document}
%
\mainmatter              % start of the contribution
%
\title{Modular Responsive Web Design using\\ Element Queries}
%
\titlerunning{Modular Responsive Web Design using Element Queries}  % abbreviated title (for running head)
%                                     also used for the TOC unless
%                                     \toctitle is used
%
\author{Lucas Wiener\inst{1}%
\and Tomas Ekholm\inst{1,2}\and Philipp Haller\inst{2}}
%
\authorrunning{Lucas Wiener et al.}   % abbreviated author list (for running head)
%
%%%% list of authors for the TOC (use if author list has to be modified)
%\tocauthor{Ivar Ekeland, Roger Temam, Jeffrey Dean, David Grove,
%Craig Chambers, Kim B. Bruce, Elisa Bertino}
%
\institute{EVRY AB, Sweden
\and KTH Royal Institute of Technology, Sweden
\mailsa\\}

\maketitle              % typeset the title of the contribution
% \index{Ekeland, Ivar} % entries for the author index
% \index{Temam, Roger}  % of the whole volume
% \index{Dean, Jeffrey}

\begin{abstract}        % give a summary of your paper
  Responsive Web Design (RWD) enables web applications to adapt to the characteristics of different devices such as screen size which is important for mobile browsing.
  Today, the only W3C standard to support this adaptability is CSS media queries.
  However, using media queries it is impossible to create applications in a modular way, because responsive elements then always depend on the global context.
  Hence, responsive elements can only be reused if the global context is exactly the same.
  This makes it extremely challenging to develop large responsive applications, because the lack of true modularity makes certain requirement changes either impossible or expensive to realize.

  In this paper we extend RWD to also include responsive modules, i.e., modules that adapt their design based on their local context, independently of the global context.
  We present the \elq{} project which implements our approach.
  \elq{} is a novel implementation of so-called \emph{element queries} which generalize CSS media queries.
  Importantly, our design conforms to existing web specifications, enabling adoption on a large scale.
  \elq{} is designed to be heavily extensible using plugins.
  Experimental results show speed-ups of the core algorithms of up to 37x compared to previous approaches.
%                         please supply keywords within your abstract
\keywords{Responsive web design, Element queries, CSS, Modularity}
\end{abstract}
%
\section{Introduction}

  Responsive Web Design (RWD) is an approach to make an application respond to the viewport size and device characteristics.
  This is currently achieved by using CSS media queries that are designed to conditionally design content by the media, such as using serif fonts when printed and sans-serif when viewed on a screen \cite{w3c_css_mq}.
  In order to reduce complexity and enable reusability, applications are typically composed of modules, i.e., interchangeable and independent parts that have a single and well-defined responsibility \cite{parnas1972criteria}.
  In order for a module to be reusable it must not assume in which context it is being used.

  In this paper we focus on the presentation layer of web applications.
  As it stands, using CSS media queries to make the presentation layer responsive precludes modularity.
  The problem is that there is no way to make a module responsive without making it context-aware, due to the fact that media queries can only target the viewport; this means that responsive modules can only respond to changes of the (global) viewport.
  Thus, a responsive module using media queries is layout dependent and has both reduced functionality and limited reusability~\cite{elq-thesis}.
  As a result, media queries can only be used for RWD of non-modular static applications.
  In a world where no better solution than media queries exists for RWD, changing the layout of a responsive application becomes a cumbersome task since it may require many responsive modules to be updated.

  \paragraph{The Problem Exemplified.}
    Imagine a company working on a big application that uses media queries for responsiveness (i.e., each responsive module assumes to have a specific percentage of the viewport size).
    The ability to change is desired by both developers and stakeholders, but is limited by this responsive approach.
    The requirement of changing a menu from being a horizontal menu at the top to being a vertical menu on the side implies that all responsive modules break, since the assumed proportionality of each module is changed.
    Even worse, if the menu is also supposed to hide on user input, the responsiveness of the module breaks, since the layout changes dynamically.
    The latter requirement is impossible to satisfy in a modular way without element queries.

  \paragraph{Requirements.}\label{sec:reqs}
    The desired behavior of a responsive module is having its inner
    design respond to the size of its \emph{container} instead of the
    viewport, to make it independent of its layout context.  The W3C
    has discussed such a feature under the name of \emph{element
      queries} given its analogy to media queries~\cite{w3c_eq_mail}.
    We have identified the following requirements: a solution must (a)
    provide the possibility for an element to automatically respond to
    changes of its parent's properties; (b) conform to the syntax of
    HTML, CSS, and JavaScript to retain the compatibility of tools,
    libraries and existing projects; (c) have adequate performance;
    (d) enable developers to write encapsulated style rules, so that
    responsive modules may be arbitrarily composed without any
    conflicting style rules.

  \paragraph{Contributions.}
    This paper makes the following contributions:
    \begin{itemize}
      \item A new design for element queries that enables responsive modules while conforming to the syntax of HTML, CSS, and JavaScript.
      \item
        Our approach is the first to enable nested elements that are responsive in a modular way, i.e., modules fully encapsulate any styling required for RWD.
        As a side effect, responsive modules may also be arbitrarily styled with CSS independent of their context.
      \item
        A new implementation\footnote{ELQ, an open-source library (MIT license): \url{https://github.com/elqteam/elq}} that offers substantially higher performance than previous approaches.
        The implementation batch-processes DOM operations in order to avoid layout thrashing (i.e., forcing the layout engine to perform multiple independent layouts).
      \item
        A run-time cycle detection system that detects and breaks cycles stemming from cyclic rules due to unrestricted usage of element queries~\cite{elq-thesis}.
    \end{itemize}

  \comment{
  \noindent
  The rest of the paper is organized as follows.
  Section~\ref{sec:elq} introduces \elq{} and its API from a user's perspective.
  Section~\ref{sec:imp} provides an overview of the implementation of \elq{}'s element resize detection system.
  In Section~\ref{sec:eval} we evaluate the performance of \elq{} and report on case studies.
  Section~\ref{sec:discussion} discusses limitations of \elq{} and related libraries, as well as the current state of standardization of element queries.
  Section~\ref{sec:related} relates \elq{} to prior work, and Section~\ref{sec:conclusion} concludes.}

\section{Conclusion}\label{sec:conclusion}
  %Responsive Web Design (RWD) enables web applications to adapt to the characteristics of different devices, which is achieved using CSS media queries.
  %However, using media queries it is impossible to create responsive applications in a modular way, because responsive elements then always depend on the global context.

  This paper extends Responsive Web Design (RWD) with {\em responsive
    modules} through element queries.  Our approach is the first to
  enable nested elements that are responsive in a modular way, i.e.,
  modules fully encapsulate any styling required for RWD.  Our
  implementation, \elq{}, is fully compatible with existing web
  standards and technologies. The element resize detection of \elq{}
  performs up to 37x better than previous algorithms.  We present a
  case study which shows that changing only about 0.6\% of the LOC is
  sufficient to enable the use of the popular Bootstrap framework in
  responsive modules.  We also report on first commercial usage of
  \elq{}.

  \comment{
  We believe \elq{} is an important contribution to realizing a
  modular form of element queries, in particular since standardization
  bodies like the RICG do not intend to standardize a complete
  solution.  In the future we intend to improve \elq{} by using
  forthcoming standards developed by the RICG to avoid some current
  limitations.}

%
% ---- Bibliography ----
%
\bibliographystyle{abbrv}
\bibliography{elq}  % elq.bib is the name of the Bibliography in this case
%
\end{document}
