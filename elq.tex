\documentclass{sigchi}

% Use this command to override the default ACM copyright statement
% (e.g. for preprints).  Consult the conference website for the
% camera-ready copyright statement.


%% EXAMPLE BEGIN -- HOW TO OVERRIDE THE DEFAULT COPYRIGHT STRIP -- (July 22, 2013 - Paul Baumann)
% \toappear{Permission to make digital or hard copies of all or part of this work for personal or classroom use is      granted without fee provided that copies are not made or distributed for profit or commercial advantage and that copies bear this notice and the full citation on the first page. Copyrights for components of this work owned by others than ACM must be honored. Abstracting with credit is permitted. To copy otherwise, or republish, to post on servers or to redistribute to lists, requires prior specific permission and/or a fee. Request permissions from permissions@acm.org. \\
% {\emph{CHI'14}}, April 26--May 1, 2014, Toronto, Canada. \\
% Copyright \copyright~2014 ACM ISBN/14/04...\$15.00. \\
% DOI string from ACM form confirmation}
%% EXAMPLE END -- HOW TO OVERRIDE THE DEFAULT COPYRIGHT STRIP -- (July 22, 2013 - Paul Baumann)


% Arabic page numbers for submission.  Remove this line to eliminate
% page numbers for the camera ready copy 

%\pagenumbering{arabic}

% Load basic packages
\usepackage{balance}  % to better equalize the last page
\usepackage{graphics} % for EPS, load graphicx instead 
%\usepackage[T1]{fontenc}
\usepackage{txfonts}
\usepackage{times}    % comment if you want LaTeX's default font
\usepackage[pdftex]{hyperref}
% \usepackage{url}      % llt: nicely formatted URLs
\usepackage{color}
\usepackage{textcomp}
\usepackage{booktabs}
\usepackage{ccicons}
\usepackage{todonotes}

% llt: Define a global style for URLs, rather that the default one
\makeatletter
\def\url@leostyle{%
  \@ifundefined{selectfont}{\def\UrlFont{\sf}}{\def\UrlFont{\small\bf\ttfamily}}}
\makeatother
\urlstyle{leo}

% To make various LaTeX processors do the right thing with page size.
\def\pprw{8.5in}
\def\pprh{11in}
\special{papersize=\pprw,\pprh}
\setlength{\paperwidth}{\pprw}
\setlength{\paperheight}{\pprh}
\setlength{\pdfpagewidth}{\pprw}
\setlength{\pdfpageheight}{\pprh}

\newcommand{\notetitle}{Allowing Responsive Web Modules}

% Make sure hyperref comes last of your loaded packages, to give it a
% fighting chance of not being over-written, since its job is to
% redefine many LaTeX commands.
\definecolor{linkColor}{RGB}{6,125,233}
\hypersetup{%
  pdftitle={\notetitle},
  pdfauthor={LaTeX},
  pdfkeywords={SIGCHI, proceedings, archival format},
  bookmarksnumbered,
  pdfstartview={FitH},
  colorlinks,
  citecolor=black,
  filecolor=black,
  linkcolor=black,
  urlcolor=linkColor,
  breaklinks=true,
}

% create a shortcut to typeset table headings
% \newcommand\tabhead[1]{\small\textbf{#1}}

% End of preamble. Here it comes the document.
\begin{document}

\title{\notetitle}

\numberofauthors{3}
\author{%
  \alignauthor{1st Author Name\\
    \affaddr{Affiliation}\\
    \affaddr{City, Country}\\
    \email{e-mail address}}\\
  \alignauthor{2nd Author Name\\
    \affaddr{Affiliation}\\
    \affaddr{City, Country}\\
    \email{e-mail address}}\\
  \alignauthor{3rd Author Name\\
    \affaddr{Affiliation}\\
    \affaddr{City, Country}\\
    \email{e-mail address}}\\
}

\maketitle

\begin{abstract}
  UPDATED---\today. This sample paper describes the
  formatting requirements for SIGCHI conference proceedings, and
  offers recommendations on writing for the worldwide SIGCHI
  readership. Please review this document even if you have submitted
  to SIGCHI conferences before, as some format details have changed
  relative to previous years. Abstracts should be about 150 words and
  are required.
\end{abstract}

\keywords{Authors' choice; of terms; separated; by semi\-colons;
  commas, within terms only; this section is required.}

\category{H.5.m.}{Information Interfaces and Presentation
  (e.g. HCI)}{Miscellaneous} \category{See
  \url{http://acm.org/about/class/1998/} for the full list of ACM
  classifiers. This section is required.}{}{}

\section{Introduction}
  \begin{itemize}
    \item Why modules? Reusability (even across applications), reduced code complexity.
    \item Why responsive design?
    \item Responsive Modules of today need to be context aware (thus, not very reusable [they only work in a specific layout]).
    \item What do we want and why? Modules that are responsive relative to its outer frame.
  \end{itemize}

  A module is an interchangeable and independent part of a program that typically has a single and well-defined responsibility.
  Modular programming is a technique to reduce complexity and enable reusability.
  In order for a module to be reusable it must not assume in which context it is being used.

  Responsive Web Design (RWD) is an approach to make the application design respond to the viewport size, in order to support varying devices.
  This is achieved by using CSS media queries to define conditional style rules.

  The problem is that there is no way to make a module responsive without it being context-aware, due to media queries only being able to target the viewport.
  Thus, a responsive module using element queries is layout dependent and has therefore limited reusability.

  The desired behavior of a responsive module is having its inner design responding to the size of \emph{its frame} instead of the viewport.
  Only then is a responsive module independent of its layout context.

  This can be achieved with the theoretical feature \emph{element queries} that enables conditional CSS rules by an arbitrary element size.
  This note presents a novel implementation of element queries in JavaScript, and discusses the new possibilities of GUI design.

\section{Examples of Broken RWD Today}
  \begin{itemize}
    \item MQ is not the solution to RWD. (MQ was not designed for RWD as the feature was released long before RWD)
    \item All elements adapt their inner design by the viewport width.
    \item Menu Example shows how MQ are broken.
    \item Limitations of MQ regarding font-size (em).
  \end{itemize}

  Media queries were designed to enable developers to conditionally design content by the media, such as using serif fonts when printed and sans-serif when viewed on a screen.
  Therefore, it is only applicable for RWD of non-modular static applications.
  In a world where no better solution than media queries exists for RWD, changing the layout of a responsive application become a cumbersome task.
  
  Imagine an application that displays the current weather of various cities as widgets, by using a weather widget module.
  The module should be responsive, so that more information such as a temperature graph over time is displayed when the widget is big.
  When the widget is small it should only display the current temperature.
  Users should also be able to add, remove and resize widgets.

  Such application cannot be built with media queries.
  Since the widgets can have varying size, the module cannot change design by breakpoints relative to the viewport.
  To overcome this problem we must change the application so that widgets always have the same size.
  If we also assume that all widgets always have a fixed percentage, for example 25\%, of the viewport width, it is possible to make the module responsive by using media queries.

  The problem now is that we have removed the reusability of the weather module, since it may only be used in applications that grant it 25\% of the viewport width.
  Also, if we decide to add a vertical menu to the application we need to change the media queries of the module.
  In more complex applications such change might result in changing hundreds of media queries.
  Even worse, if the menu is supposed to hide on user input the responsiveness of the module breaks since the layout changes dynamically.

  As we can see, even with limited requirements there still are significant flaws with using media queries for responsive modules.

\section{A Solution}
  \begin{itemize}
    \item Parents should decide the layout of their children, and the children should adapt their inner design accordingly.
    \item Valid language syntaxes (HTML, CSS, JS).
  \end{itemize}

\section{Why is a Native Implementation troublesome?}
  \begin{itemize}
    \item Performance issues.
    \item Cite Tab Atkins of RICG (he states that it is infeasible to standardize this).
  \end{itemize}

\section{A JavaScript Implementation}
  \begin{itemize}
    \item Why is this pragmatic? Compatability, no impact (performance, language) on apps that do not need responsive modules.
    \item Satisfies the requirements for a solution given above.
    \item Present Elq's API.
    \item Present the performance.
    \item Note drawbacks (but only drawbacks for added functionality!).
  \end{itemize}

\section{Discussion and Summary of Related Work}
  \begin{itemize}
    \item Performance, APIs, Features.
    \item The mirror functionality of Elq makes it uniquely suitable for nested modules.
  \end{itemize}

\section{Conclusion}
  \begin{itemize}
    \item Production ready.
    \item Probably no standard (or not in a long time).
  \end{itemize}

\section{Acknowledgments}

Sample text: We thank all the volunteers, and all publications support
and staff, who wrote and provided helpful comments on previous
versions of this document. Authors 1, 2, and 3 gratefully acknowledge
the grant from NSF (\#1234--2012--ABC). \textit{This whole paragraph is
  just an example.}

% Balancing columns in a ref list is a bit of a pain because you
% either use a hack like flushend or balance, or manually insert
% a column break.  http://www.tex.ac.uk/cgi-bin/texfaq2html?label=balance
% multicols doesn't work because we're already in two-column mode,
% and flushend isn't awesome, so I choose balance.  See this
% for more info: http://cs.brown.edu/system/software/latex/doc/balance.pdf
%
% Note that in a perfect world balance wants to be in the first
% column of the last page.
%
% If balance doesn't work for you, you can remove that and
% hard-code a column break into the bbl file right before you
% submit:
%
% http://stackoverflow.com/questions/2149854/how-to-manually-equalize-columns-
% in-an-ieee-paper-if-using-bibtex
%
% Or, just remove \balance and give up on balancing the last page.
%
% \balance{}

% REFERENCES FORMAT
% References must be the same font size as other body text.
\bibliographystyle{SIGCHI-Reference-Format}
\bibliography{sample}

\end{document}

%%% Local Variables:
%%% mode: latex
%%% TeX-master: t
%%% End:
