\documentclass{sigchi}

% Use this command to override the default ACM copyright statement
% (e.g. for preprints).  Consult the conference website for the
% camera-ready copyright statement.


%% EXAMPLE BEGIN -- HOW TO OVERRIDE THE DEFAULT COPYRIGHT STRIP -- (July 22, 2013 - Paul Baumann)
% \toappear{Permission to make digital or hard copies of all or part of this work for personal or classroom use is      granted without fee provided that copies are not made or distributed for profit or commercial advantage and that copies bear this notice and the full citation on the first page. Copyrights for components of this work owned by others than ACM must be honored. Abstracting with credit is permitted. To copy otherwise, or republish, to post on servers or to redistribute to lists, requires prior specific permission and/or a fee. Request permissions from permissions@acm.org. \\
% {\emph{CHI'14}}, April 26--May 1, 2014, Toronto, Canada. \\
% Copyright \copyright~2014 ACM ISBN/14/04...\$15.00. \\
% DOI string from ACM form confirmation}
%% EXAMPLE END -- HOW TO OVERRIDE THE DEFAULT COPYRIGHT STRIP -- (July 22, 2013 - Paul Baumann)


% Arabic page numbers for submission.  Remove this line to eliminate
% page numbers for the camera ready copy 

%\pagenumbering{arabic}

% Load basic packages
\usepackage{balance}  % to better equalize the last page
\usepackage{graphics} % for EPS, load graphicx instead 
%\usepackage[T1]{fontenc}
\usepackage{txfonts}
\usepackage{times}    % comment if you want LaTeX's default font
\usepackage[pdftex]{hyperref}
% \usepackage{url}      % llt: nicely formatted URLs
\usepackage{color}
\usepackage{textcomp}
\usepackage{booktabs}
\usepackage{ccicons}
\usepackage{todonotes}

% llt: Define a global style for URLs, rather that the default one
\makeatletter
\def\url@leostyle{%
  \@ifundefined{selectfont}{\def\UrlFont{\sf}}{\def\UrlFont{\small\bf\ttfamily}}}
\makeatother
\urlstyle{leo}

% To make various LaTeX processors do the right thing with page size.
\def\pprw{8.5in}
\def\pprh{11in}
\special{papersize=\pprw,\pprh}
\setlength{\paperwidth}{\pprw}
\setlength{\paperheight}{\pprh}
\setlength{\pdfpagewidth}{\pprw}
\setlength{\pdfpageheight}{\pprh}

\newcommand{\notetitle}{Allowing Responsive Web Modules}

% Make sure hyperref comes last of your loaded packages, to give it a
% fighting chance of not being over-written, since its job is to
% redefine many LaTeX commands.
\definecolor{linkColor}{RGB}{6,125,233}
\hypersetup{%
  pdftitle={\notetitle},
  pdfauthor={LaTeX},
  pdfkeywords={SIGCHI, proceedings, archival format},
  bookmarksnumbered,
  pdfstartview={FitH},
  colorlinks,
  citecolor=black,
  filecolor=black,
  linkcolor=black,
  urlcolor=linkColor,
  breaklinks=true,
}

% create a shortcut to typeset table headings
% \newcommand\tabhead[1]{\small\textbf{#1}}

% End of preamble. Here it comes the document.
\begin{document}

\title{\notetitle}

\numberofauthors{3}
\author{%
  \alignauthor{1st Author Name\\
    \affaddr{Affiliation}\\
    \affaddr{City, Country}\\
    \email{e-mail address}}\\
  \alignauthor{2nd Author Name\\
    \affaddr{Affiliation}\\
    \affaddr{City, Country}\\
    \email{e-mail address}}\\
  \alignauthor{3rd Author Name\\
    \affaddr{Affiliation}\\
    \affaddr{City, Country}\\
    \email{e-mail address}}\\
}

\maketitle

\begin{abstract}
  UPDATED---\today. This sample paper describes the
  formatting requirements for SIGCHI conference proceedings, and
  offers recommendations on writing for the worldwide SIGCHI
  readership. Please review this document even if you have submitted
  to SIGCHI conferences before, as some format details have changed
  relative to previous years. Abstracts should be about 150 words and
  are required.
\end{abstract}

\keywords{Authors' choice; of terms; separated; by semi\-colons;
  commas, within terms only; this section is required.}

\category{H.5.m.}{Information Interfaces and Presentation
  (e.g. HCI)}{Miscellaneous} \category{See
  \url{http://acm.org/about/class/1998/} for the full list of ACM
  classifiers. This section is required.}{}{}

\section{Introduction}
  \begin{itemize}
    \item Why modules? Reusability (even across applications), reduced code complexity.
    \item Why responsive design?
    \item Responsive Modules of today need to be context aware (thus, not very reusable [they only work in a specific layout]).
    \item What do we want and why? Modules that are responsive relative to its outer frame.
  \end{itemize}

\section{Examples of Broken RWD Today}
  \begin{itemize}
    \item MQ is not the solution to RWD. ($MQ -> RWD$. $RWD !-> MQ$)
    \item All elements adapt their inner design by the viewport width.
    \item Menu Example shows how MQ are broken.
    \item Limitations of MQ regarding font-size (em).
  \end{itemize}

\section{A Solution}
  \begin{itemize}
    \item Parents should decide the layout of their children, and the children should adapt their inner design accordingly.
    \item Valid language syntaxes (HTML, CSS, JS).
  \end{itemize}

\section{Why is a Native Implementation troublesome?}
  \begin{itemize}
    \item Performance issues.
    \item Cite Tab Atkins of RICG (he states that it is infeasible to standardize this).
  \end{itemize}

\section{A JavaScript Implementation}
  \begin{itemize}
    \item Why is this pragmatic? Compatability, no impact (performance, language) on apps that do not need responsive modules.
    \item Satisfies the requirements for a solution given above.
    \item Present Elq's API.
    \item Present the performance.
    \item Note drawbacks (but only drawbacks for added functionality!).
  \end{itemize}

\section{Discussion and Summary of Related Work}
  \begin{itemize}
    \item Performance, APIs, Features.
    \item The mirror functionality of Elq makes it uniquely suitable for nested modules.
  \end{itemize}

\section{Conclusion}
  \begin{itemize}
    \item Production ready.
    \item Probably no standard (or not in a long time).
  \end{itemize}

\section{Acknowledgments}

Sample text: We thank all the volunteers, and all publications support
and staff, who wrote and provided helpful comments on previous
versions of this document. Authors 1, 2, and 3 gratefully acknowledge
the grant from NSF (\#1234--2012--ABC). \textit{This whole paragraph is
  just an example.}

% Balancing columns in a ref list is a bit of a pain because you
% either use a hack like flushend or balance, or manually insert
% a column break.  http://www.tex.ac.uk/cgi-bin/texfaq2html?label=balance
% multicols doesn't work because we're already in two-column mode,
% and flushend isn't awesome, so I choose balance.  See this
% for more info: http://cs.brown.edu/system/software/latex/doc/balance.pdf
%
% Note that in a perfect world balance wants to be in the first
% column of the last page.
%
% If balance doesn't work for you, you can remove that and
% hard-code a column break into the bbl file right before you
% submit:
%
% http://stackoverflow.com/questions/2149854/how-to-manually-equalize-columns-
% in-an-ieee-paper-if-using-bibtex
%
% Or, just remove \balance and give up on balancing the last page.
%
\balance{}

\section{References Format}
Your references should be published materials accessible to the
public. Internal technical reports may be cited only if they are
easily accessible (i.e., you provide the address for obtaining the
report within your citation) and may be obtained by any reader for a
nominal fee. Proprietary information may not be cited. Private
communications should be acknowledged in the main text, not referenced
(e.g., ``[Golovchinsky, personal communication]'').

Use a numbered list of references at the end of the article, ordered
alphabetically by first author, and referenced by numbers in
brackets~\cite{ethics,Klemmer:2002:WSC:503376.503378}. For papers from
conference proceedings, include the title of the paper and an
abbreviated name of the conference (e.g., for Interact 2003
proceedings, use Proc.\ Interact 2003). Do not include the location of
the conference or the exact date; do include the page numbers if
available. See the examples of citations at the end of this document
and in the accompanying \texttt{BibTeX} document.

References \textit{must be the same font size as other body
  text}. References should be in alphabetical order by last name of
first author. Example reference formatting for individual journal
articles~\cite{ethics}, articles in conference
proceedings~\cite{Klemmer:2002:WSC:503376.503378},
books~\cite{Schwartz:1995:GBF}, theses~\cite{sutherland:sketchpad},
book chapters~\cite{winner:politics}, a journal issue~\cite{kaye:puc},
websites~\cite{acm_categories,cavender:writing},
tweets~\cite{CHINOSAUR:venue}, patents~\cite{heilig:sensorama}, and
online videos~\cite{psy:gangnam} is given here. This formatting is a
slightly abbreviated version of the format automatically generated by
the ACM Digital Library (\url{http://dl.acm.org}) as ``ACM Ref''. More
details of reference formatting are available at:
\url{http://www.acm.org/publications/submissions/latex_style}.

% REFERENCES FORMAT
% References must be the same font size as other body text.
\bibliographystyle{SIGCHI-Reference-Format}
\bibliography{sample}

\end{document}

%%% Local Variables:
%%% mode: latex
%%% TeX-master: t
%%% End:
